\chapter{Conclusions and perspectives}\label{chp:conclusion}


This thesis presented the latest available results from two separate searches for new physics beyond the standard model with the CMS detector at the LHC. One is searching for heavy resonances in dielectron final state, another is searching new physics in top quark production in dielectron or dimuon with some jets/bjets final states. The strategies of these two analyses are different. One is directly searching for the localized excess in the dielectron mass spectrum using a as simple and robust as possible way. The another one is indirectly searching for new physics and using a dedicated multivariate analysis to separate tW and \ttbar~ processes to make the analysis more sensitive to new physics.

The search for new heavy resonances decaying in the dielectron final state was described in details in Chapter \ref{chap:Zprime}. The data used are the ones collected by the CMS experiment in 2016 with 35.9 \fbinv and the ones in 2017 with 41.4 \fbinv. The event selection is optimized in order to be highly efficient for high-energy electrons/positrons and to avoid loosing potential signal events. The main source of background, the Drell-Yan process, is estimated from simulation. Data-driven approach is used for validating the subleading background processes which are estimated from simulations, also it is used for the determination of the background coming from quantum chromo-dynamics process. After having inspected the dielectron invariant mass, no significant excess over the standard model background is observed, and upper limits at 95\% confidence level are set on the ratio of production cross-section times branching ratio of a new resonance to the one at the Z boson peak, using a Bayesian approach. With the measured upper limits on the cross-section ratio, lower limits on the resonance masses have been set for particles predicted by various models. In particular, for spin 1 resonances, masses below 4.7 TeV, for the \ZPSSM particle from the sequential standard model, and below 4.1 TeV for the superstring inspired \ZPPSI particle could be excluded with the combination of dielectron (using 35.9 \fbinv from 2016 $+$ 41.4 \fbinv from 2017) and dimuon (using 36.3 \fbinv from 2016) channels. This is the CMS most stringent limits to-date on the topic. 
The ATLAS collaboration has also provided several publications on the same topic. For instance, ATLAS \cite{ATLAS-CONF-2017-027} puts a 95\% confidence level lower mass limits of 4.5 TeV for the \ZPSSM model and 3.8 TeV for the \ZPPSI one after combining both dielectron and dimuon final states using 36.1 \fbinv data. These results are well in agreement with the ones obtained by CMS and showed in this thesis. Recently ATLAS published their full Run2 analysis (see Ref. \cite{Aad:2019fac}), and found the following 95\% confidence level lower mass limits: 5.1 TeV for the \ZPSSM model and 4.5 TeV for the \ZPPSI.


The second analysis presented in this thesis is the search for new physics in top quark production and was described in details in Chapter \ref{chap:tW}. The single top quark production in association with a W boson is probed together with the top quark pair production to find the new physics signatures. Due to the similarity of the final states for tW and \ttbar~ processes, a dedicated multivariate analysis is used to separate these two processes. The data used is collected by the CMS experiment in 2016 with 35.9 \fbinv. Using the ee and \mumu final states and combining with e$\mu$ final state, the observed $95\%$ CL limit band on effective couplings are found to be [$-1.01$,0.70] for \CG, [$-0.41$,0.17]  for \CtG, [$-0.96$,5.74]  for \CtW, [$-3.82$,0.63] for \Cphiq, [$-0.22$,0.22] for \CuG and [$-0.46$,0.46] for \CcG. The extracted values give the first experimental bound on the \CG coupling and improve upon limits previously obtained at 8 TeV for \CtG. The limits obtained on the \CtW, \Cphiq, \CuG and \CcG couplings from the tW process are complementary to the limits from the single top t-channel process.

Although we haven't find the new physics until now, we may find it in the future with the help of much more data from LHC Run 3 and ``high luminosity LHC'' (HL-HLC). The foreseen operating scenario for Run 3 is to reach 13 - 14 TeV pp collision energy and keep the instantaneous luminosity at the level of $10^{34}~\mathrm{cm^{-2}s^{-1}}$. At the end of Run 3 (the end of 2023), the LHC will deliver $\sim$ 300 \fbinv data which is 2 times larger than what we have now. From 2026 on, the HL-LHC starts and it is expected to operate at an enhanced luminosity of $5\times10^{35} \mathrm{cm^{-2}s^{-1}}$. The total integrated luminosity delivered at the end of HL-LHC will be $\sim$ 3000 \fbinv which increases the statistics by around one order of magnitude comparing to the total delivered luminosity at the end of Run 3. Form a detector point of view, there will be many challenges for a detector to work very well at the future HL-LHC. For instance, due to the much higher radiation from the increased instantaneous luminosity, there is a plan to replace the full tracker and the endcap calorimeter of the ECAL subdetector with a high granularity calorimeter made of silicon sensors and tungsten absorber. Besides, the average number of pileup interactions will be much increased which is around 4 times higher what it is now. Therefore, the trigger system should be carefully optimized to save the interesting events. From physics analysis point of view, the results about SM parameter measurements can be updated precisely, for example the Higgs properties measurement. In addition, some rare SM processes (e.g. four top production, $\mathrm{H}\rightarrow\mu\mu$) could be observed thanks to the much increased statistics. Last but not least, using the massive cumulated data the new physics can be thorough searched at TeV energy scale, and a basic question about whether or not any new physics exists at the TeV energy scale can be addressed by the LHC experiments.



