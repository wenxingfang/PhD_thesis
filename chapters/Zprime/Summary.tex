\clearpage
\section{Summary}
\label{sec:Zprime_summary}


A search for narrow resonances in dielectron invariant mass spectra has been performed using data recorded by CMS in 2016 and 2017 from proton-proton collisions at $\sqrt{s} = 13$ TeV. The results of the analysis have been also combined with those of the analogous search in the dimuon final state.
The integrated luminosity for the dielectron sample is 35.9 \fbinv\ in 2016 and 41.4 \fbinv\ in 2017, for the dimuon sample it is 36.3 \fbinv\ . Observations are in agreement with standard model expectations. Upper limits at 95\% confidence level on the parameter of interest $R_{\sigma}$ have been derived.
 
Limits are set on the masses of various hypothetical particles. For the $\ZPSSM$ particle,
which arises in the sequential standard model, and for the superstring-inspired $\ZPPSI$ particle,
95\% confidence level lower mass limits for the combined channels are found to be 4.70
and 4.10 TeV, respectively. These limits extend the previous ones from CMS by 1.3 TeV in
both models. The corresponding limits for Kaluza-Klein gravitons arising in the Randall-Sundrum
model of extra dimensions with coupling parameters $\overline{M}_{Pl}$ of 0.01, 0.05, and 0.10 are 2.10, 3.65, and 4.25 TeV, respectively. The limits extend previous published CMS
results by 0.6 (1.1) TeV for a $\overline{M}_{Pl}$ value of 0.01 (0.10).

My personal contributions to the analysis include performing regular checks of the detector response by providing invariant mass
plots and HEEP selection efficiency using both data and simulations, and on the extraction of the data to simulation scale factor during the data taking periods. Because the HEEP ID scale factor is the vital part of this analysis, many studies and checks have been performed to make sure we have understand it. The HEEP ID is also used in other CMS analyses, so providing HEEP ID scale factor becoming one of my EPR work.
Besides, I was responsible for the studies on the mass scale and resolution of the ECAL detector, which are key inputs for the computation of the limits, given that they are used to define the signal model. Moreover, I was responsible for a study of the electromagnetic calorimeter saturation effects although we don't have saturated events in our final mass distribution in data. In addition, I was worked on the fit to the background contribution and estimated the various systematics uncertainties of the analysis. Last but not least I was responsible of providing the final mass spectra.

This results in several publications by the CMS collaboration: two PAS documents, one corresponding to the analysis of 35.9 \fbinv\ of integrated luminosity collected in full 2016 \cite{CMS-PAS-EXO-16-047} and the other corresponding to the analysis of 41.4 \fbinv\ of integrated luminosity collected in full 2017 \cite{CMS-PAS-EXO-18-006}. There are one paper \cite{Sirunyan2018} where the results obtained analyzing the full 2016 dataset with 35.9 \fbinv\ of integrated luminosity.

The ATLAS collaboration has also provided several publications on the same topic and the latest available results from ATLAS \cite{ATLAS-CONF-2017-027} put a 95\% confidence level lower mass limit of 4.5 TeV for the $\ZPSSM$ model and 3.8 TeV for the $\ZPPSI$ one after combining both dielectron and dimuon final states using 36.1 \fbinv\ data. These results are well in agreement with the ones obtained by CMS and showed in this chapter.  