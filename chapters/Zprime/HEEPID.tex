\section{Object and Event Selection}\label{sec:Zprime_HEEP}
Electrons are required to pass the HEEP selection which is listed in Table \ref{tab:HEEPV70} and then are combined to form dielectron candidates.
If more than one dielectron candidate is found in the event, only the pair with the two largest electron $E_T$ is retained.
There is no charge requirement for dielectron candidates and this is made to avoid efficiency losses at high mass for the main analysis.
Besides, at least one of the electron candidates has to be in barrel (events with both electron candidates in endcaps regions are rejected).
Events in data are required to satisfy the trigger selection described in Section \ref{sec:Zprime_trigger}.
MC events are weighted using turn on curves shown in figures \ref{fig:L1_eff_2017}, \ref{fig:HLT_turnon_2016} and \ref{fig:HLT_turnon_2017} for considering the L1 or HLT effects.


\begin{table}[!h]
  \begin{center}
\smallskip\noindent
\resizebox{\linewidth}{!}{%
    \begin{tabular}{lll}
      \hline
      Variable                          & Barrel                             & Endcap                             \\
      \hline
      \multicolumn{3}{c}{Acceptance selections}\\
      \ET                             & \ET$>$ 35 GeV                    & \ET$>$ 35 GeV                    \\
      $\eta$                            & $|\eta| < 1.4442$             & $1.566 < |\eta| < 2.5$        \\
      \hline
      \multicolumn{3}{c}{Identification selections}\\
      %\texttt{isEcalDriven}             & true                               & true                               \\
      $\Delta\eta_{in}^{seed}$          & $|\Delta\eta_{in}^{seed}| < 0.004$ & $|\Delta\eta_{in}^{seed}| < 0.006$ \\
      $\Delta\phi_{in}$                 & $|\Delta\phi_{in}| < 0.06$         & $|\Delta\phi_{in}| < 0.06$         \\
      $H/E$                             & $H/E < 1/E + 0.05$                 & $H/E < 5/E + 0.05$                 \\
      $\sigma_{i\eta i\eta}$            & -                                  & $\sigma_{i\eta i\eta} < 0.03$      \\
      $\frac{\EOnexFive}{\EFive}$ and $\frac{\ETwoxFive}{\EFive}$        & $\frac{\EOnexFive}{\EFive}>0.83$ or $\frac{\ETwoxFive}{\EFive}>0.94$ & -                              \\
      Inner lost layer hits             & lost hits $\le 1$                  & lost hits $\le 1$                  \\
      Impact parameter $d_{xy}$        & $|d_{xy}|<0.02$ cm                   & $|d_{xy}|<0.05$ cm                   \\
      \hline
      \multicolumn{3}{c}{Isolation selections}\\
      \small{Calorimeter isolation} $Iso$                  & $Iso < 2 + 0.03\ET[\mathrm{GeV}] + 0.28\rho$     & $Iso < 2.5 + 0.28\rho$ (\ET<50 GeV) \\
                                        &                                    & else $Iso < 2.5 + 0.03(\ET[\mathrm{GeV}]-50) + 0.28\rho$ \\
      $p_T$ isolation $Isopt$          & $Isopt <$ 5 GeV                   & $Isopt <$ 5 GeV                   \\
      \hline
    \end{tabular}}
    \caption{Definitions of HEEP selection cuts.}
    \label{tab:HEEPV70}
  \end{center}
\end{table}
