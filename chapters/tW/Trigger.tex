
\section{Triggers}
\label{tW_Trigger}

In this analysis we use various sets of triggers in order to achieve an optimal selection efficiency.
For each dilepton channel (ee, $\mu\mu$) we take the logical OR of the dilepton triggers listed
in Table \ref{tab:trigger}.
The increase of the instantaneous luminosity delivered by the LHC in Run H needed the (re)introduction of the DZ cut for pure dimuon trigger.

We complement the partially inefficient dilepton triggers by single lepton triggers and remove the overlap between two
primary data sets by vetoing events that fired a dilepton trigger in the single lepton primary
data sets.
%Because the e$\mu$ channel is complemented by two single lepton primary data sets, the
%procedure has two steps: We first add events from the SingleElectron primary data set that
%fired single electron triggers but failed the double lepton triggers that we use to select events
%in the MuonEG primary data set. We furthermore add events from the SingleMuon primary
%data set that fired single muon triggers but fired none of the triggers we use to select events in
%either the MuonEG or the SingleElectron primary data set.
Table \ref{tab:trigger} gives an overview of all triggers used in this analysis.
Using \ttbar ~and tW MC samples, it is found that by adding single lepton triggers, trigger efficiency is increased by around 5\%.


% Please add the following required packages to your document preamble:
%\usepackage{multirow}
\begin{table}[th]
\centering
\begin{tabular}{lll}
\hline
channel               & path                                                      & dataset             \\ \hline \hline
\multirow{2}{*}{ee}   & HLT\_Ele23\_Ele12\_caloIdL\_TrackIdL\_IsoVL\_DZ           & data \& MC          \\
                      & HLT\_Ele27\_WPTight\_Gsf                                  & data \& MC          \\ \hline
%\multirow{7}{*}{e$\mu$}  & HLT\_Mu8\_TrkIsoVVL\_Ele23\_CaloIdL\_TrackIdL\_IsoVL      & data runs B-G \& MC \\
%                      & HLT\_Mu23\_TrkIsoVVL\_Ele12\_CaloIdL\_TrackIdL\_IsoVL     & data runs B-G \& MC \\
%                      & HLT\_Mu8\_TrkIsoVVL\_Ele23\_CaloIdL\_TrackIdL\_IsoVL\_DZ  & data only run H \\
%                      & HLT\_Mu23\_TrkIsoVVL\_Ele12\_CaloIdL\_TrackIdL\_IsoVL\_DZ & data only run H \\
%                      & HLT\_Ele27\_WPTight\_Gsf                                  & data \& MC          \\
%                      & HLT\_IsoMu24                                              & data \& MC          \\
%                      & HLT\_IsoTkMu24                                            & data \& MC          \\ \hline
\multirow{6}{*}{$\mu\mu$} & HLT\_Mu17\_TrkIsoVVL\_Mu8\_TrkIsoVVL                      & data runs B-G \& MC \\
                      & HLT\_Mu17\_TrkIsoVVL\_TkMu8\_TrkIsoVVL                    & data runs B-G \& MC \\
                      & HLT\_Mu17\_TrkIsoVVL\_Mu8\_TrkIsoVVL\_DZ                  & data only run H     \\
                      & HLT\_Mu17\_TrkIsoVVL\_TkMu8\_TrkIsoVVL\_DZ                & data only run H     \\
                      & HLT\_IsoMu24                                              & data \& MC          \\
                      & HLT\_IsoTkMu24                                            & data \& MC          \\ \hline \hline
\end{tabular}
\caption{Summary of the signal triggers}
\label{tab:trigger}
\end{table}


Since the trigger information is available in simulated samples, we require FullSim MC samples to fire the trigger and finally apply corrections for any data/MC disagreement using scale factors.
TOP group recommended trigger scale factor (described in \cite{trSF}) are used.
We also have found trigger scale factors as a cross checked which gives similar result (although they are not used in this analysis).
