\chapter*{Introduction}

What is the origin of matter? What are the most fundamental elements in our universe? What are the main forces between matter?
These are interesting, basic, and important questions. Although they are difficult to be answered, we are on the way to find the answers.

Particle physics is a subject that tries to find the basic particles in the Universe and to understand the interaction mechanisms between these fundamental particles. The most well known and successful theory in particle physics is the Standard Model (SM) which managed to explain until now all the experimental observations with outstanding precision. The final missing piece of the SM, the ``Higgs'' particle which was introduced in 1964 by Brout, Englert, and Higgs in order to explain the origin of masses of elementary particles, has been discovered in 2012 (48 years after its prediction) at a mass of around 125 GeV by the ATLAS and CMS experiments at the Large Hadron Collider (LHC).

The LHC is the largest hadron collider in the world providing proton-proton collisions with the highest center-of-mass energy ever achieved (13 TeV from 2015 to 2018). There are four main experiments at the LHC, two of them CMS and ATLAS are general purpose detectors. The discovery of the Higgs boson by joint efforts of the ATLAS and CMS collaborations is one of the most important achievements of modern particle physics research and accomplished one of the main goals of the LHC program.

Nevertheless, besides the tremendous successes of the SM, it is not able to describe the full picture of Nature. Indeed, it does not show candidates of dark matter and dark energy, it does not predict the oscillation of neutrinos, it does not have a good explanation for the asymmetry between matter and antimatter. It has in addition some issues of internal consistency, such as the hierarchy problem, a large number of free parameters and so on. Therefore, the SM is generally considered as an effective theory of a more fundamental theory at high energy. In order to address some shortcomings of the SM, several models that go beyond the SM have been proposed, such as supersymmetric models (SUSY), which provide a candidate of dark matter and provide an explanation to the Higgs mass fine-tuning problem, or The Grand Unified Theory (GUT) which tries to unify electromagnetic, weak, and strong interactions into one interaction through extensions of the SM gauge group, or the large extra dimensions theory, which involves additional spatial dimensions to explain the weakness of the gravitational force compared to the other forces. These beyond SM models typically introduce new neutral bosons heavier than the standard model Z boson, which are generically called \ZP bosons.

If the mass scale of such new particles are reachable in collider experiment, these particles would manifest themselves as a localised excess of events in the observed invariant mass spectra. In this thesis, direct search for new heavy resonances decaying into the dielectron final state has been performed using the CMS detector. This channel has the advantage that electrons can be reconstructed and identified with high efficiency which leads to a low background contamination coming from misreconstructed electron candidates. Besides, the main component of SM background in this channel is Drell-Yan process, which is well understood and its rate is small in the high mass region. These facts give a strong motivation for searching for new heavy resonances in the dielectron final state. In this thesis, the analysis of data collected by the CMS experiment during years 2016 and 2017 are reported.

However, if the new physics scale is not reachable at the LHC, new physics could affect SM interactions indirectly, through modifications of SM couplings or enhancements of rare SM processes. Due to its large mass, close to the electroweak symmetry breaking scale, the top quark is expected to play an important role in several new physics scenarios. An effective field theory (EFT) approach which is a model-independent approach is used in this thesis to search for new physics in the top quark sector in the dilepton (ee, \mumu) final states using the data collected by CMS in 2016.


The thesis is organised as follows. The SM of particle physics is introduced in Chapter \ref{sec:SM}, including a description of fundamental particles, the forces between these particles, the main properties of the SM Drell-Yan process and an introduction to the EFT theory. Chapter \ref{chp:BSM} lists the shortcomings of the SM and addresses how various theories beyond the SM propose to solve them. In particular, the models that predict additional massive resonances are introduced. The motivations for searching for new physics in the top quark sector is also given. Chapter \ref{chap:LHC_CMS} presents an introduction to the LHC machine, including its design and operational parameters, as well as the phenomenological aspects of the proton-proton interactions. The CMS detector is also introduced in detail in this chapter. The reconstruction of the different particles produced in the proton proton collisions in CMS is explained in Chapter \ref{chap:Event_reconstruction}. Chapter \ref{chap:Zprime} describes in detail the results of the search for new resonances decaying into the dielectron final state and all the aspects of the analysis are covered. The results of the search for new physics in the top quark sector are shown in Chapter \ref{chap:tW}, covering as well all the aspects of the analysis. Finally, Chapter \ref{chp:conclusion} exposes the conclusions coming from both searches.


